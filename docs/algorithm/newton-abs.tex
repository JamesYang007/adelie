\subsection{Newton with Adaptive Bisection Starts (Newton-ABS)}\label{ssec:newton-abs}

In this section, we improve the Newton's Method described in~\Cref{ssec:newton}.
Specifically, we first discuss how to find lower and upper bounds $h_\star$, $h^\star \in [0,\infty)$,
respectively, such that the root lies in $[h_\star, h^\star]$.
We then discuss an adaptive bisection method that preceeds the Newton's Method
to find a clever initial point $h^{(0)}$.

We first begin with the derivation of $h_\star$ and $h^\star$.
Note that the root-finding problem for $\varphi$
is equivalent to finding the largest value $h > 0$ such that $\varphi(h) \geq 0$, or
\begin{align*}
    \sum\limits_{i=1}^p
    \frac{v_i^2}{(D_{ii} h + \lambda)^2}
    \geq
    1
\end{align*}
Let $a(h), b(h) \in \R^p$ be defined by $a_k(h) := D_{kk} h + \lambda$
and $b_k(h) := \frac{\abs{v_k}}{a_k(h)}$ for each $k=1,\ldots, p$.
Then, by Cauchy-Schwarz,
\begin{align*}
    \norm{v}_1
    &:=
    \sum\limits_{k} \abs{v_k}
    =
    \sum\limits_{k} a_k(h) b_k(h)
    \leq
    \norm{a(h)}_{2} \norm{b(h)}_2
\end{align*}
Hence, if $\norm{a(h)}_2 \leq \norm{v}_1$,
then $\varphi(h) \geq 0$.
We see that $\norm{a(h)}_2 \leq \norm{v}_1$ if and only if
\begin{align*}
    \sum\limits_{i=1}^p
    (D_{ii} h + \lambda)^2
    \leq
    \norm{v}_1^2
\end{align*}
This inequality can be solved for $h$ using the quadratic formula.
Let $\tilde{h}$ be the solution.
Then, we have a potentially tighter lower bound $h_\star := \max(\tilde{h}, 0)$ than zero.

Next, we discuss $h^\star$.
Similar to $h_\star$, we wish to find the smallest $h$ such that
\begin{align*}
    \sum\limits_{i=1}^p
    \frac{v_i^2}{(D_{ii} h + \lambda)^2}
    \leq
    1
\end{align*}
Approximating this problem, since
\begin{align}
    \sum\limits_{i=1}^p
    \frac{v_i^2}{(D_{ii} h + \lambda)^2}
    &=
    \sum\limits_{i: D_{ii} = 0}
    \frac{v_i^2}{\lambda^2}
    +
    \sum\limits_{i: D_{ii} > 0}
    \frac{v_i^2}{(D_{ii} h + \lambda)^2}
    \nonumber\\&\leq 
    \sum\limits_{i: D_{ii} = 0}
    \frac{v_i^2}{\lambda^2}
    +
    h^{-2}
    \sum\limits_{i: D_{ii} > 0}
    \frac{v_i^2}{D_{ii}^2 }
    \label{eq:nmab:upper-approx}
\end{align}
by setting 
\begin{align*}
    h^\star
    := 
    \sqrt{
        \frac{
            \sum\limits_{i: D_{ii} > 0} \frac{v_i^2}{D_{ii}^2}
        }{
            1 - \lambda^{-2} \sum\limits_{i : D_{ii} = 0} v_i^2
        }
    }
\end{align*}
we have that $\varphi(h^\star) \leq 0$.
This shows that the root must lie in $[h_\star, h^\star]$.

In practice, $\varphi$ may decay incredibly rapidly near $h \approx 0$ and have an extremely flat tail.
To protect against slow convergence of Newton's Method in this case,
we may use $h_\star$ and $h^\star$ to first perform a bisection method to find the initial starting point.
The key idea is to use bisection first for a few iterations 
to avoid the region of slow changes in the Newton iterations.\todojames{refer to a figure?}
Then, once sufficiently close to the root, we apply the Newton's Method for the fast guaranteed convergence.
Although one may use a simple bisection method of splitting the interval $[h_\star, h^\star]$ in halves
until sufficiently small, we propose an \emph{adaptive bisection method} 
that can significantly reduce the number of bisections.

We describe the adaptive bisection method.
Ideally, we would like to know if the root is closer to $h_\star$ or $h^\star$.
If we believe that the root is much closer to $h_\star$,
we do not have to bisect $[h_\star, h^\star]$ at the mid-point, 
but perhaps at a point closer to $h_\star$.
Likewise, if the root were much closer to $h^\star$, 
we would like to bisect at a point closer to $h^\star$.
Since we do not know the root, we would like to quantify a \emph{prior} of 
the root being closer to $h_\star$.
With this in mind, note that in~(\ref{eq:nmab:upper-approx}), we 
approximated the problem of finding the smallest $h$ such that $\varphi(h) \leq 0$
by solving the problem for an upper bound of $\varphi$.
Then, the more accurate the approximation, the closer the approximate solution $h^\star$ is to the root.
The approximation essentially came from using the fact that $D_{ii} h^\star \leq D_{ii} h^\star + \lambda$
for $D_{ii} > 0$.
This motivates us to consider the worst approximation error (rate)
\begin{align*}
    w
    := 
    \max\limits_{i: D_{ii} > 0} \frac{\lambda}{D_{ii} h^\star + \lambda}
    =
    \frac{\lambda}{D_\star h^\star + \lambda}
    \in 
    (0,1)
\end{align*}
where $D_\star := \min\limits_{i : D_{ii} > 0} D_{ii}$.
If $w$ is small, the approximation in~(\ref{eq:nmab:upper-approx}) is tight,
which implies that the root is close to $h^\star$.
Hence, $1-w$ represents the prior that the root is close to $h^\star$.
We bisect at the new point $h := wh_\star + (1-w)h^\star$.
If $\varphi(h) \geq 0$, we use $h^{(0)} := h$ as the initial point for the Newton Method.
Otherwise, we set $h^\star := h$ and repeat the argument.
\Cref{alg:nabs:ab} summarizes this procedure.
Combining \Cref{alg:nabs:ab} with Newton's method,
we have \Cref{alg:nabs:nabs},
which we call \emph{Newton with Adaptive Bisection Starts} (Newton-ABS).

\begin{algorithm}
    \caption{Adaptive Bisection}\label{alg:nabs:ab}
    \KwData{$D$, $v$, $\lambda$, $\varepsilon$}
    compute $h_\star$ that solves
    $
        \sum\limits_{i=1}^p
        (D_{ii} h + \lambda)^2
        \leq
        \norm{v}_1^2
    $ and take the positive part\;
    $
        h^\star
        \gets
        \sqrt{
            \frac{
                \sum\limits_{i: D_{ii} > 0} \frac{v_i^2}{D_{ii}^2}
            }{
                1 - \lambda^{-2} \sum\limits_{i : D_{ii} = 0} v_i^2            
            }
        }
    $\;
    $D_\star \gets \min\limits_{i : D_{ii} > 0} D_{ii}$\;
    $w \gets \frac{\lambda}{D_\star h^\star + \lambda}$\;
    $h \gets wh_\star + (1-w)h^\star$\;
    \While{$\varphi(h) < 0$ and $\abs{\varphi(h)} > \varepsilon$}{
        $h^\star \gets h$\; 
        $w \gets \frac{\lambda}{D_\star h^\star + \lambda}$\;
        $h \gets wh_\star + (1-w)h^\star$\;
    }
    return $h$;
\end{algorithm}

\begin{algorithm}
    \caption{Newton with Adaptive Bisection Starts (Newton-ABS)}\label{alg:nabs:nabs}
    \KwData{$D$, $v$, $\lambda$, $\varepsilon$}
    $h \gets$ result of \Cref{alg:nabs:ab}\;
    \If{$\varphi(h) \geq 0$}{
        $\tilde{h} \gets$ result of Newton's Method starting at $h^{(0)} = h$\;
    }
    $\beta^\star \gets (D + \lambda \tilde{h}^{-1} I)^{-1} v$\;
    return $\beta^\star$\;
\end{algorithm}

Algorithm~\ref{alg:nabs:nabs} can be further optimized.
For example, if $h^\star - h_\star$ is below some threshold (e.g. $0.1$),
then we may skip bisection entirely and start Newton's Method at $h^{(0)} = h_\star$.
This is because the adaptive bisection may move too slowly if the range is too small.
From experimentation, this happens relatively often.
On a similar note, one may also enforce enough movement towards $h_\star$
by taking the max of $w$ with a minimal probability (e.g. $0.05$).
This will ensure that at least some proportion of $h_\star$ is taken 
if the prior is too strongly suggestive that the root is close to $h^\star$.
The idea is that it is always better to overshoot towards $h_\star$ such that $\varphi$ is non-negative,
so that Newton's Method can quickly converge down to the root,
than to slowly bisect to the root.

Similar to our strategy in~\Cref{alg:nabs:ab},
there are other existing methods that try to achieve adaptive bisection.
A popular algorithm is the Brent's method.
Although Brent's method performs extremely well for this task,
as seen in~\Cref{fig:bench:pgd-newton},
Newton-ABS is still superior.
We refer to the full benchmark comparison to~\Cref{ssec:benchmark:pgd-newton}.
